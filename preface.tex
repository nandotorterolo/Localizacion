%This is the Preface
%%=========================================
\addcontentsline{toc}{section}{Prefacio}
\section*{Prefacio}

En las redes sociales los usuarios, comparten su localización, sea por medio del GPS del dispositivo móvil o directamente declarando que viven en determinado país, ciudad o pueblo. Sin embargo, cuando estos datos no existen, es posible localizarlo en función de información de contexto. 

Se denomina homofilia, a la tendencia de las personas de relacionarse con otras que se parecen a ellas. Estos similitudes pueden ser respecto a muchos atributos, educación, edad, clase social, religión, entre otros.

Los usuarios suelen compartir características de sus amigos, suele publicar sus preferencias, si simpatiza por algún equipo de fútbol relacionado a un país, podemos asumir con cierto grado de precisión que vive en ese país. 

Construir modelos predictivos que permitan determinar la localización o nacionalidad de un usuario es el objetivo de este proyecto.


\begin{flushright}
Montevideo, 2016-08-11 \\[1pc]
\end{flushright}

